\chapter{Related Work\label{chap5}}

%\section{Related Work}

This chapter will cover a brief introduction of related research work that has focused on virtualization solutions
for real-time applications by developing new or bringing real-time capabilities to existing virtualization
solutions. Furthermore, it presents some of research efforts made to improve real-time responsiveness of real-time guests.


%This chapter will cover the work related from the domain of embedded virtual setups and mechanisms used to lower interrupt latency.
%% Schild et al. evaluated performance of hardware-assisted virtualization extensions... does not comply with our setup
%% Aichouch et al. has observed RTOS latenices in virtualized setup for Litmus-RT... they noticed increased wcet for RTOS

\section{KVM-based solutions}
Kernel-based Virtual Machine (KVM) is an open source type-2 hypervisor extensively used as virtualization solution in enterprise domain.
KVM uses existing Linux kernel mechanisms like scheduling policies etc. for virtual machines. 
Kiszka \cite{kiszka2009towards} has proposed real-time improvements that includes using PREEMPT\_RT patch to add real-time capabilities to host Linux kernel.
Zhang et al. presented a real-time virtualization solution using KVM hypervisor on x86 platform \cite{zuo2010performance}.
Their solution consisted of an RTOS and a GPOS running in separate virtual machines. 
They applied two performance improvement mechanisms on host Linux kernel: CPU shielding and interrupt prioritization.
The performance tunings significantly improved worst-case interrupt response time of the RTOS guest.
Furthermore, they showed worst-case Process Dispatch Latency Time (PDLT) of PREEMPT\_RT patched Linux guest improved after using the performance tunings.

\section{Xen-based solutions}
Xen is a widely used open source type-1 hypervisor \cite{Barham:2003:XAV:1165389.945462}. 
Many research projects has been conducted so far to make Xen appropriate for real-time virtualization.
The split driver model for IO devices makes it challenging to use for real-time workloads \cite{Yoo:2011:TUI:2103799.2103816}.
Masrur et al. \cite{masrur2010vm} evaluated real-time performance of Xen SEDF scheduler and proposed PSEDF scheduler to separate
real-time domains from others. The showed that separating real-time domain from others can lower interrupt latencies and jitters.
Xi et al. has developed RT-Xen framework to support real-time scheduling policies in Xen \cite{Xi:2011:RTR:2038642.2038651}.
RT-Xen project extends Xen by appending new scheduling widely used in real-time domain.

\section{Microkernel-Based Solutions}
Schild et al. \cite{schild2009faithful} used x86 virtualization extensions to run unmodified guest OS on L4/Fiasco microkernel.
They observed increased interrupt response times for real-time guest interrupts and showed that in certain cases the interrupt latency
can increase significantly.
Bruns et al. investigated real-time performance of L4/Fiasco microkernel \cite{bruns2010evaluation}.
The measured interrupt latencies and context-switch times of real-time guest and used FreeRTOS as reference.
The experiments showed that average execution times and interrupt latencies for real-time guest are significantly larger than real-time kernel.
They observed cache contention as the main contributor to the virtualization overhead.


\section{Other solutions}
XtratuM is type-1 non-preemptive hypervisor that is designed specifically for real-time safety critical applications \cite{Carrascosa:2014:XHR:2668138.2668142}.
XtratuM deploys para-virtualization to achieve minimality. It provides minimalistic set of hypercalls to guests for hardware interfacing.
It achieves deterministic performance by using static partitioning of resources allocated to each virtual machine.
The approach is very close to the one used by Phidias hypervisor.

Quest-V is an open source separation kernel that uses hardware assisted-virtualization to isolate guest in time and space \cite{West:2016:VSK:2966277.2935748}.
It supports applications of mixed-criticality by sandboxing a real-time guest partition on multicore processor.
Quest-V uses static partitioning of hardware resources across guests to avoid traditional virtualization overheads.
It allows real-time guest OS to receive IO interrupts without hypervisor intervention and tries to avoid vmexits as much as possible.

Gordon et al. proposed exit-less interrupt (ELI) delivery mechanism to remove vmexits in situations where hypervisor intervention is not required \cite{Gordon:2012:EBP:2150976.2151020}.
Their solution is software-only where a shadow-IDT is maintained by the host to inject virtual interrupts to the unmodified guest.
They have showed ELI approach improves throughput of IO intensive workloads.


%%Bruns et. al evaluated mickro-kernel based virtualization solutions. They observed 30% to 50% increase in execution times of the tasks.
%% they pointed cache contentions as the main reason of this overhead and predicted jitter might be more in processors with bigger caches

%% IDEA measure cache hit and miss penalties for given parameters of xeon processor using online tool!!!

%%Yoo et al. developed MobiVMM, a hypervisor for mobile phones [REF].
%%XtratuM is a type-1 hypervisor which is designed for safety critical avionics embedded systems. It runss on LEON3 SPARC V8 processor.
%%Kim et al. proposed two cache management techniques for real-time worloads: virtual LLC and virtual coloring [REF].
%%The techiqnue allow temporal isolation of real-time tasks running in a virtual machine. 
%%They implemented proposed scheme on KVM for x86 and ARM and proved significant utilization benifit.

