\chapter{Virtualization Overhead\label{chap3}}

This chapter presents an analysis of virtualization overhead for Phidias hypervisor and records 
results of experiments conducted on the x86 platform. 
In order to evaluate virtualization overhead of the x86 processor, a hypothetical mixed criticality system is used
that includes an RTOS and GPOS running concurrently in separate virtual machines.
The key parameter used to measure the overhead is real-time event handling latency, also known as 
interrupt response time (IRT).
The interrupt latency is the time that elapses between the arrival of an external interrupt request (IRQ) and execution of event handler starts.
Figure \ref{fig-latency-keymetric} depicts the difference in interrupt latency experienced by external interrupt request in native and virtualized environment.
In a virtual setup, the latency is expected to increase due to hypervisor intervention when an external interrupt is routed to the guest through the hypervisor.
The hypervisor injects external interrupts to owner guest by servicing interrupt first. Execution of hypervisor before the interrupt is passed to 
the owner guest increases the interrupt latency.
It is important to measure virtualization overhead to determine if deadlines of the real-time tasks of guest RTOS can still be respected.

\begin{figure}[!htb]
\begin{center}
\begin{tikzpicture}

\newcommand\yh{0.65cm}

%\draw[step=1cm, gray, very thin, dotted] (0,0) grid (15,6);

\node at (0,4) [rectangle, draw=black, thick, fill=Blue!20, rounded corners, minimum height = \yh, minimum width = 2.3cm, anchor=south west] (kubefore) 
				{kernel/user};
\node at (2.4,4) [rectangle, draw=black, thick, fill=orange!30, rounded corners, minimum height = \yh, minimum width = 0.9cm, anchor=south west] (hardware) 
				{\small{HW}};
\node at (3.4,4) [rectangle, draw=black, thick, fill=orange!30, rounded corners, minimum height = \yh, minimum width = 1.8cm, anchor=south west] (kservice) {kernel};
\node at (5.4,4) [rectangle, draw=black, thick, fill=OliveGreen!20, rounded corners, minimum height = \yh, minimum width = 3cm, anchor=south west] (kuafter) {\small{IRQ} handler};
\draw[black, thick] (2.35,3.25) -- (2.35,5.25) node [above] {IRQ};
\draw[black, thick] (5.3,3.25) -- (5.3,5.25) node [above] {};
\draw[black, dashed, <->] (2.35,3.6) -- (5.3,3.6) node [below, align=center, midway] {\scriptsize{interrupt latency}};

\draw[black, thick, ->] (0,3.8) -- (10,3.8) node at (10,3.8) [below] {t};

\node at (0,1) [rectangle, draw=black, thick, fill=Blue!20, rounded corners, minimum height = \yh, minimum width = 2.3cm, anchor=south west] (kubefore) {kernel/user};
\node at (2.4,1) [rectangle, draw=black, thick, fill=orange!30, rounded corners, minimum height = \yh, minimum width = 0.9cm, anchor=south west] (hardware) 
				{\small{HW}};

\node at (3.4,1) [rectangle, draw=black, thick, fill=red!20, rounded corners, minimum height = \yh, minimum width = 4cm, anchor=south west] (hyper) {hypervisor};

\node at (7.5,1) [rectangle, draw=black, thick, fill=orange!30, rounded corners, minimum height = \yh, minimum width = 2cm, anchor=south west] (kservice) {kernel};
\node at (9.7,1) [rectangle, draw=black, thick, fill=OliveGreen!20, rounded corners, minimum height = \yh, minimum width = 3cm, anchor=south west] (kuafter) {\small{IRQ} handler};

\draw[black, thick] (2.35,0.25) -- (2.35, 2.25)  node [left] {IRQ};
%\draw[black, thick] (7.4, 0.65) -- (7.4, 2.25) node [above] {};
\draw[black, thick] (9.6,0.25) -- (9.6, 2.25) node [above] {};

\draw[black, dashed, |<->|] (3.35,1.9) -- (7.4,1.9) node [above, align=center, midway] {\scriptsize{virtualization overhead}};
\draw[black, dashed, <->] (2.35,0.6) -- (9.6,0.6) node [below, align=center, midway] {\scriptsize{interrupt latency}};


\draw[black, thick, ->] (0,0.8) -- (13.5,0.8) node at (13.5,0.8) [below] {t};

\end{tikzpicture}
\end{center}
\ifreport
\caption{Interrupt Latency in native and virtualized environment}
\fi
\label{fig-latency-keymetric}
\end{figure}

Hypervisor overhead consists of context switch times, execution of hypervisor code to identify and inject virtual interrupt to the real-time guest (rt-guest) and
resource contention. The resource contention overhead arises due to shared resources between guests and hypervisor.
While this chapter focuses on analyzing and measuring components of virtualization overhead, the next chapter covers techniques to lower overhead components.

\section{Experimental Setup}
All experiments are conducted on Intel Xeon E5-2603v4 processor \cite{intel-ark-xeon} mounted on ASUS X99E WS motherboard \cite{asus-x99e}. 
The processor runs at $1.7GHz$ clock speed and had access to a $4GB$ RAM memory.
This processor contains $6$ physical CPU cores where each core has private L1 and L2 caches, the cache capacities are $64KB$ and $512KB$ respectively.
The last-level cache (L3) is shared between cores and has a $15MB$ capacity. 
The processor supports Intel VT-x technology. The virtualization extensions on Xeon processor includes virtualization of LAPIC and MMU.

The evaluated virtualization solution contains hypervisor and two guests: an RTOS and a GPOS.
RTOS guest has been created by applying latest PREEMPT\_RT patchset (v4.11.12-rt15) to Linux kernel (v4.11.12).
The details of PREEMPT\_RT patch are covered in section \ref{sec:rtos-ext-gpos}.
The GPOS guest is Linux kernel (v4.11.8).
A small patchset was applied to both real-time and general purpose Linux operating systems, given in Appendix \ref{app:a}.
Linux kernel uses PIT (Programmable Interval Timer) to calculate processor clock frequency during bootup process.
Phidias hypervisor does not support virtualization of PIT device at the moment, the patch is used to fix processor clock frequencies to known values.


An external device on PCIe (Peripheral Component Interconnect Express) bus manufactured by VersaLogic is used to generate external events \cite{versalogic}.
The device is equipped with EXAR XR17v354 SoC \cite{xr17v354} and supports up to 12 general purpose input/outputs (GPIOs). 
The VersaLogic GPIO device has the capability to send Message Signaled Interrupt (MSI) to the processor when an external event is detected.
Each GPIO can be programmed to cause an external interrupt when a rising pulse is detected on the inputs.
The RTOS guest was given full access to the device by granting read/write accesses to the device memory.
Furthermore, real-time guest was given full access to HPET (High Precision Event Timer) device. 
The HPET device is needed by cyclictest\cite{cyclictest}, a widely used tool to evaluate real-time performance of PREEMPT\_RT Linux kernel.
The cyclictest has been used for two purposes. First, it was used to validate latency results produced for gpio device interrupts.
Secondly, it was used as a load to exercise POSIX thread operations as explained in next section.
Table \ref{experiment-setup} summarizes setup used to conduct experiments.


\begin{table}[!htbp]
\centering
%\begin{longtable}{|p{.20\textwidth} | p{.70\textwidth}|}
\begin{tabular}{|p{2cm}|p{2cm}|p{6cm}|p{2cm}|}  
\hline
\textbf{Resource} & \textbf{CPUs} & \textbf{Devices} & \textbf{Physical Memory} \\ \hline 
RTOS & & & \\ 
(rt-guest) & one vCPU & vLAPIC, vIOAPIC, vLAPIC-Timer, vUART, PCIe GPIO, HPET & 256MB \\ \hline
GPOS & & & \\ 
(gp-guest) & one vCPU & vLAPIC, vIOAPIC, vLAPIC-Timer, vUART & 256MB \\ \hline
PHIDIAS & & & \\ 
Hypervisor & two cores & LAPIC, IOAPIC, LAPIC Timer, UART & -- \\ \hline
Hardware & \multicolumn{3}{|c|}{Xeon  E5-2603v4 Processor} \\ \hline
\end{tabular}
\caption{Experiment Setup}
\label{experiment-setup}
\end{table}

\section{Measurement Setup} \label{sec:measurement-setup}
The interrupt latency measurement setup consists of an external FPGA board \cite{terasic-de0nano} and external PC for latency measurement.
Two GPIOs of FPGA are connected to PCIe GPIO device. 
First, to cause an interrupt and send a low-to-high pulse and second to listen for the
response time of GPIO interrupt handler. External PC sends commands to FPGA on UART to generate an interrupt.
The measurement setup generates an interrupt for rt-guest and expects rt-guest to respond by asserting an output GPIO.

A small patch for EXAR XR17v354 SoC was added to rt-guest to configure GPIOs for desired behavior.
And a separate kernel module was designed and loaded on rt-guest to register interrupt handler for the PCIe GPIO device.
Source code for added gpio device functionality and interrupt handler module is given in Appendix \ref{app:a}.
The very first instruction executed by gpio interrupt handler is the assertion of an output pin, to mark time instance when interrupt handler starts execution.
The pin assertion marks the response time of gpio interrupt handler.
It allows external setup to measure latency by calculating the time difference between observed response time and interrupt generation.
The FPGA hardware is programmed to perform a routine as follows:
\begin {itemize}
	\item {wait to receive a command from external PC on UART interface}
	\item {generate an interrupt for rt-guest by asserting input GPIO of PCIe device}
	\item {count FPGA clock cycles while waiting for rt-guest interrupt handler to send a response on output GPIO}
	\item {send counted clock cycles to external PC on UART interface}
\end {itemize}

Figure \ref{fig-measure-steup} describes measurement setup and demonstrates how GPIOs are used to measure latency.
The external PC calculates interrupt latency by scaling the count value by known FPGA clock frequency (50MHz). 
A small script on external PC can be used to repeat measurement routine multiple time to find worst-case interrupt latency.


\begin{figure}[!htb]
\begin{center}
\begin{tikzpicture}

%\draw[step=1cm, gray, very thin, dotted] (0,0) grid (10,8);

\node at (0,0) [rectangle, draw=black, thick, fill=Yellow!20, rounded corners, minimum height = 2cm, minimum width = 2.0cm, align=center, text width=1.7cm, anchor=south west] (pc) 
		{External\\ PC};

\node at (3,0) [rectangle, draw=black, thick, fill=Yellow!20, rounded corners, minimum height = 2cm, minimum width = 2cm, anchor=south west] (fpga) 
		{FPGA};

\node at (7,0) [rectangle, draw=black, thick, fill=RoyalBlue!20, rounded corners, minimum height = 2cm, minimum width = 2cm, align=center, text width=1.7cm, anchor=south west] (pciedev) 
		{PCIe GPIO\\ Device};

\node at (6.5,3) [rectangle, draw=black, thick, fill=ForestGreen!20, rounded corners, minimum height = 2cm, minimum width = 3cm, align=center, text width=1.7cm, anchor=south west] (rtg) {};
\node[below right, inner sep=5pt] at (rtg.north west) {rt-guest};

\node at (7,3.2) [rectangle, draw=black, thick, fill=RoyalBlue!20, rounded corners, minimum height = 0.75cm, minimum width = 2cm, align=center, text width=1.7cm, anchor=south west] (irqhandler) 
		{IRQ handler};

\draw[black, thick, ->] (irqhandler.south) -- (pciedev);
\draw[black, thick, ->] ([xshift=-0.2cm]pciedev.north) -- ([xshift=-0.2cm]rtg.south) ;


\draw[black, thick, <->] (2,1) -- (3,1) node [below, midway, align=center] {\tiny{UART}};
\draw[black, thick, ->] (5,1.5) -- (7,1.5)  node [below, midway, align=center] {\tiny{input gpio}};
\draw[black, thick, <-] (5,0.5) -- (7,0.5)  node [below, midway, align=center] {\tiny{output gpio}};

\draw[black, rounded corners, thick] (1.5,5) -- (3,5) node [above, midway, align=center] {\tiny{input gpio}} -- (3,6) -- (5.5,6);
\draw[black, rounded corners, thick] (1.5,4) -- (4.5,4)  node [above left, midway, align=center] {\tiny{output gpio}} -- (4.5,5) -- (5.5,5);

\draw[black, thick, dotted] (3,3.5) -- (3,6.25);
\draw[black, thick, dotted] (4.5,3.5) -- (4.5,6.25);

\node (latency) at (3.75,3.75) [align=center, text width=1.5cm] {latency};
\draw[black, thick, ->] (2.5,3.75) -- (3,3.75);
\draw[black, thick, <-] (4.5,3.75) -- (5,3.75);

\begin {scope} [opacity=0.25]
\node at (1.25,3.25) [thick, rectangle, draw=black, thick, fill=white, rounded corners, minimum height = 3.5cm, minimum width = 4.5cm, anchor=south west] (lbox) {};
\draw [anchor=south west] {(6.5,1) ellipse (0.25cm and 0.75cm)};
\draw[black, thick, dotted] (1.25,3.25) -- (6.5,0.25);
\draw[black, thick, dotted] (5.75,3.25) -- (6.5,1.75);
\end {scope}


%\node at (1.25,3.25) [thick, ellipse, draw=black, thick, fill=white, rounded corners, minimum height = 1cm, minimum width = 0.5cm, anchor=south west] (gpios) {};

%\draw[black, thick, loosely dotted] (1.5,5) -- (3,5);

\end{tikzpicture}
\end{center}
\ifreport
\caption{Latency measurement setup}
\fi
\label{fig-measure-steup}
\end{figure}



\section{Methodology} \label{sec:methodology}
The latency overhead of Phidias hypervisor is measured in two steps. 
At first, the real-time guest has been executed bare-metal, without a hypervisor. 
In this scenario, real-time guest (rt-guest) executes natively with full control over hardware resources. 
The interrupt latency in this case purely consists of rt-kernel overhead to schedule interrupt handler.
The second step consists of running the same rt-guest in a virtualized environment.
Now Phidias has control over hardware resources. 
This time the interrupt latency consists of rt-guest kernel overhead and 
virtualization overhead. The virtualization overhead includes hypervisor runtime to route
the interrupt to rt-guest and contention of shared resources. 
%The real-time guest is assigned to a physical core to remove hypervisor overhead and CPU contention. 
The virtualization overhead is depicted in Figure \ref{fig-latency-keymetric}.
In both setups the latency measurement mechanism consisting of external PC, FPGA and PCIe GPIO device is same.

\begin{table}[!htb]
\centering
%\begin{longtable}{|p{.20\textwidth} | p{.70\textwidth}|}
\begin{tabular}{|r|p{8cm}|}  
\hline
& \\
\textbf{Microbenchmark} & \textbf{Description} \\ \hline 
\mcachepressure{} & The benchmark processes big arrays of data to add pressure on the caches. The sizes of arrays are bigger than L2 cache.\\ \hline
\mforkops{} 	& The benchmark exercises fork operation of Linux kernel from userspace\\ \hline
\mfileops{} 	& The benchmark exercises file operation of Linux kernel from userspace \\ \hline
\mhackbench{} 	& The benchmark adds load on kernel scheduler. The load was added using hackbench\tablefootnote{hackbench -p -s1024 -l10000}. \\ \hline
\mmmapops{} 	& The benchmark exercises memory map operations of Linux kernel from userspace \\ \hline
\mstdout{} 		& The benchmark exercises data output operations of Linux kernel on the null device from userspace \\ \hline
\mthreadops{} 	& The benchmark exercises POSIX thread operations from userspace. The load was added using cyclictest\tablefootnote{cyclictest -a -t10 -p80 -n -l100000 -i1000 -q}. \\ \hline
\mnoload{} 	& no load was added \\ \hline
\end{tabular}
\caption{Microbenchmarks used as synthetic workloads} % needs to go inside longtable environment
\label{microbenchmarks}
\end{table}


Synthetic microbenchmarks were used to create realistic workload scenarios to increase the reliability of the results. 
These benchmarks include small programs that exercise common operations typically used by the userspace applications.
These benchmarks add load on real-time guest while measuring the interrupt latency.
A brief description of each microbenchmark is included in Table \ref{microbenchmarks}.
The latency measurement procedure is repeated after loading rt-guest with each benchmark.
The \mcachepressure{} benchmark is executed from rt-guest to add load on private L1 and L2 caches
shared between the hypervisor and the rt-guest.
While latency measurement experiments are repeated for rt-guest with different loads the gp-guest 
is running idle. 
The intention is to measure overhead mainly introduced by sharing of resources between rt-guest and the hypervisor.
The LLC is shared between gp-guest and rt-guest, and will be polluted by gp-guest during the experiments.
However, as gp-guest is left running in idle mode the cache pollution from gp-guest is expected to be small.
Next chapter records virtualization overhead when shared LLC is polluted by gp-guest.
Rest of the benchmarks exercise kernel library operations to add load on the rt-guest kernel.
Different load scenarios are investigated to record variations in interrupt latency and virtualization overhead.

%Notice that cyclictest 
%\footnote{Cyclictest is a tool widely used to evaluate real-time performance of PREEMPT\_RT patch.} %\\
%\url{https://wiki.linuxfoundation.org/realtime/documentation/howto/tools/cyclictest}} 
%and hackbench 
%\footnote{Hackbench is benchmark commonly used to stress kernel scheduler.}
%are used to add load on the rt-guest kernel. 

The measurement setup is capable to calculate interrupt latency very precisely.
The worst-case interrupt latency is computed by repeating latency measurement procedure 
10 thousand times. The interrupts are generated periodically at the rate of approximately 2ms.

The measurement procedure consists of running PREEMPT\_RT guest (rt-guest) in a native and virtual setup with buildroot \footnote{\url{https://buildroot.org/}} based userspace.
A script from external PC commands FPGA to generate an interrupt on PCIe GPIO device.
The FPGA generates an interrupt and measures number of clock cycles that elapse between sending interrupt and receiving
a response from the rt-guest interrupt handler. The elapsed time, in clock cycles, is sent back external PC
where the script calculates interrupt latency and stores the result for further processing.
The script repeats the measurement procedure for ten thousand times
and reports average and worst-case event handling latency. 
The procedure is repeated with different workload conditions on rt-guest generated using microbenchmarks.


\section{Baseline}
In the interest of measuring virtualization overhead of hypervisor, a reference needs to be established. 
The most reliable way to establish a baseline is to measure real-time guest interrupt handling latency
in native setup. In native setup, rt-guest is allowed to execute without hypervisor. Figure \ref{fig-native}
demonstrates the native setup that is used to establish baseline.

\begin{figure}[!htb]
\begin{center}
\begin{tikzpicture}



\node at (0,3.325) [rectangle, draw=black, thick, fill=white, rounded corners, minimum height = 3cm, minimum width = 6cm, anchor=south west] (rtguest) {};

%\node[inner sep=0pt, text width=2cm] (rtgtext) at (1.5, 5) {PREEMPT\_RT guest};
\node[below right, inner sep=5pt, text width=2cm] at (rtguest.north west) {PREEMPT\_RT\\ Linux\\ (rt-guest)};
\node[above right, inner sep=5pt] (tux) at (rtguest.south)  {\includegraphics[width=2cm]{figures/tux.jpg}};
%\node[inner sep=0pt] (tux) at (4.5, 5)  {\includegraphics[width=2cm]{figures/tux.jpg}};

\node at (0,2.250) [rectangle, draw=black, thick, fill=white, rounded corners, minimum height = 1cm, minimum width = 6cm, anchor=south west] (core0) {Core0};

\node at (0,1.125) [rectangle, draw=black, thick, fill=white, rounded corners, minimum height = 1cm, minimum width = 2.875cm, anchor=south west] (lapic) {LAPIC};
\node at (3,1.125) [rectangle, draw=black, thick, fill=white, rounded corners, minimum height = 1cm, minimum width = 3cm, anchor=south west] (ioapic) {IOAPIC};
\node at (0,0) [rectangle, draw=black, thick, fill=white, rounded corners, minimum height = 1cm, minimum width = 2.875cm, anchor=south west] (device) {PCIe Dev.};
\node at (3,0) [rectangle, draw=black, thick, fill=white, rounded corners, minimum height = 1cm, minimum width = 3cm, anchor=south west] (uart) {UART};


\end{tikzpicture}
\end{center}
\ifreport
\caption{Native Setup used to measure baseline interrupt latency}
\fi
\label{fig-native}
\end{figure}

The latency measurement procedure described in section \ref{sec:methodology} was executed to determine worst-case interrupt latency
of the rt-guest. The experiments were repeated under different load conditions using the microbenchmarks.
\begin{figure}[!htbp]
\begin{center}
\begin{tikzpicture} %[scale=1.2]

\begin{axis} [ ybar=2pt, height=8cm, width=15cm, enlarge y limits={upper,value=0.2}, %enlargelimits=0.1,
                       legend style={at={(0.5,-0.3)}, anchor=north, legend columns=-1},
                       ylabel={interrupt latency(us)}, %title=vmexits after bootup, 
		       bar width=16pt,
       		       %%enlarge x limits={abs=0.8cm}, 		       
                       symbolic x coords={no load, cache pressure,forkops, fileops, hackbench, mmapops, stdout, threadops, all benchmarks},
                       xtick=data, nodes near coords, 
		       %nodes near coords align={vertical},
			%ymajorgrids=true, grid style = very thin,
		       every node near coord/.append style={rotate=90, anchor=west},
                       x tick label style={rotate=45,anchor=east}, ]                    

	\addplot [fill=black!20] coordinates {
				(no load, 67.6)				
				(cache pressure, 73.7)
				(forkops, 25.9)
				(fileops, 24.7)
				(hackbench, 9.3)
				(mmapops, 23.4)
				(stdout, 25.8)
				(threadops, 12.4)		
				(all benchmarks, 19.8)				
			    };

	\addplot [postaction={pattern=north east lines}] coordinates {
				(no load, 44.2)
				(cache pressure, 43.1)
				(forkops, 6.1)
				(fileops, 6.1)
				(hackbench, 4.5)
				(mmapops, 4.4)
				(stdout, 6.1)
				(threadops, 5.0)
				(all benchmarks, 4.9)		
			    };

\legend{worst-case, average}
\end{axis}

\end{tikzpicture}
\end{center}
\ifreport
\caption{Baseline Interrupt Latency}
\fi
\label{plot-baseline}
\end{figure}


The baseline worst-case and average interrupt latency histogram is plotted in Figure \ref{plot-baseline}.
The worst-case latency without load is about three times more than interrupt latency with load. 
One straightforward reason for lower worst-case latency with benchmarks is warming up of caches.
When benchmarks are running, they exercise kernel library operations that warm up the caches. 
As a result, most memory accesses are satisfied from caches when external interrupt arrives.
The average-case latency has also reduced by more than $7$ times.
On PREEMPT\_RT Linux, an interrupt handler executes in thread context and it is a schedulable entity. 
The benchmarks keep the scheduler running in a tight loop, increasing the frequency of scheduling events.
This increases the chances of the interrupt handler to be scheduled soon, 
which improves both the average and worst-case interrupt latency.
An exception to this behavior is worst-case latency with \mcachepressure{} benchmark.
This time the caches are polluted by the benchmark, thrashing most cached entries of data.
As a result, the guest code misses most of the data in caches, that increases both worst-case and average-case latency.


\section{Virtualization overhead of PHIDIAS} \label{sec:virtual-overhead}
The latency measurement procedure is repeated for the rt-guest 
executing in a virtual machine to determine virtualization overhead for Phidias hypervisor.
The virtualized setup consists of rt-guest and gp-guest assigned to separated cores of the processor.
Both guests have virtualized LAPIC, LAPIC-Timer, IOAPIC, and UART.
The guest physical memory is limited to $256MB$ and LLC is shared.
Rt-guest has direct access to  PCIe GPIO device, 
but interrupts are received by Phidias hypervisor first and then routed to the rt-guest.
Figure \ref{fig-virtual} shows virtual setup used to measure virtualization overhead.
Furthermore, the rt-guest has direct access to HPET device. 
The HPET device is used by the rt-guest kernel as clock-source.

\begin{figure}[!htb]
\begin{center}
\begin{tikzpicture}

%\node at (0,5.625) [rectangle, draw=black, thick, fill=white, rounded corners, minimum height = 3cm, minimum width = 4cm, text width=2cm, anchor=south west] (rtguest) {};

\node at (0,2.25) [rectangle, draw=black, thick, fill=white, rounded corners, minimum height = 6cm, minimum width = 5cm, anchor=south west] (rtguest) {};

%\node[inner sep=0pt, text width=3cm] (rtgtext) at (2, 8) {PREEMPT\_RT guest};
%\node[inner sep=0pt] (tux) at (3, 7)  {\includegraphics[width=1.5cm]{figures/tux.jpg}};
\node[below right, inner sep=5pt, text width=2cm] at (rtguest.north west) {PREEMPT\_RT\\ Linux\\ (rt-guest)};
\node[below left, inner sep=2pt] (tux1) at (rtguest.north east)  {\includegraphics[width=1.8cm]{figures/tux.jpg}};


\node at (0.5,5.1) [rectangle, draw=black, thick, fill=black!5, rounded corners, minimum height = 0.8cm, minimum width = 4cm, anchor=south west] (vcpu1) {vCPU};
\node at (0.5,4.2) [rectangle, draw=black, thick, fill=black!5, rounded corners, minimum height = 0.8cm, minimum width = 1.9cm, anchor=south west] (vlapic1) {vLAPIC};
\node at (2.5,4.2) [rectangle, draw=black, thick, fill=black!5, rounded corners, minimum height = 0.8cm, minimum width = 2cm, anchor=south west] (vioapic1) {vIOAPIC};
\node at (0.5,3.2) [rectangle, draw=black, thick, fill=black!5, rounded corners, minimum height = 0.8cm, minimum width = 1.9cm, anchor=south west] (vdevice) {vTimer};
\node at (2.5,3.2) [rectangle, draw=black, thick, fill=black!5, rounded corners, minimum height = 0.8cm, minimum width = 2cm, anchor=south west] (vuart1) {vUART};

\node at (0.5,2.25+0.1) [rectangle, draw=black, thick, fill=black!10, rounded corners, minimum height = 0.7cm, minimum width = 4cm, anchor=south west] (vdevice) {\small{PCIe GPIO Device}};

\node at (0,1) [rectangle, draw=black, thick, fill=black!3, rounded corners, minimum height = 1cm, minimum width = 5cm, anchor=south west] (phidias1) {PHIDIAS};
\node at (0,0-0.125) [rectangle, draw=black, thick, fill=black!3, rounded corners, minimum height = 1cm, minimum width = 5cm, anchor=south west] (core0) {Core0};


\draw[black, thick, dashed] (5.25,-0.25) -- (5.25,9); %node at (0,0) [below] {release time};

\node at (5.5,2.25) [rectangle, draw=black, thick, fill=white, rounded corners, minimum height = 6cm, minimum width = 5cm, anchor=south west] (guest) {};
%\node[inner sep=0pt, text width=3cm] (rtgtext) at (7, 8) {guest2};
%\node[inner sep=0pt] (tux) at (7.5, 7)  {\includegraphics[width=1.5cm]{figures/tux.jpg}};

\node[below right, inner sep=5pt, text width=2cm] at (guest.north west) {Linux\\ (gp-guest)};
\node[below left, inner sep=2pt] (tux2) at (guest.north east)  {\includegraphics[width=1.8cm]{figures/tux.jpg}};

\node at (6,5.1) [rectangle, draw=black, thick, fill=black!5, rounded corners, minimum height = 0.8cm, minimum width = 4cm, anchor=south west] (vcpu2) {vCPU};
\node at (6,4.2) [rectangle, draw=black, thick, fill=black!5, rounded corners, minimum height = 0.8cm, minimum width = 1.9cm, anchor=south west] (vlapic2) {vLAPIC};

\node at (8,4.2) [rectangle, draw=black, thick, fill=black!5, rounded corners, minimum height = 0.8cm, minimum width = 2cm, anchor=south west] (vioapic2) {vIOAPIC};
\node at (6,3.2) [rectangle, draw=black, thick, fill=black!5, rounded corners, minimum height = 0.8cm, minimum width = 1.9cm, anchor=south west] (vdevice) {vTimer};
\node at (8,3.2) [rectangle, draw=black, thick, fill=black!5, rounded corners, minimum height = 0.8cm, minimum width = 1.9cm, anchor=south west] (vuart2) {vUART};

\node at (5.5,1) [rectangle, draw=black, thick, fill=black!3, rounded corners, minimum height = 1cm, minimum width = 5cm, anchor=south west] (phidias2) {PHIDIAS};
\node at (5.5,0-0.125) [rectangle, draw=black, thick, fill=black!3, rounded corners, minimum height = 1cm, minimum width = 5cm, anchor=south west] (core1) {Core1};


\draw[black, thick, dotted] (-0.25, 2.1) -- (11,2.1);

\end{tikzpicture}
\end{center}
\ifreport
\caption{Virtual Setup used to measure virtualization overhead}
\fi
\label{fig-virtual}
\end{figure}

%%\begin{figure}[!htb]
\begin{center}
\begin{tikzpicture}

\begin{axis} [ ybar=2pt, height=8cm, width=15cm, enlarge y limits={upper,value=0.2}, %enlargelimits=0.1,
                       legend style={at={(0.5,-0.3)}, anchor=north, legend columns=-1}, 
                       ylabel={interrupt latency ($\mu{}s$)}, %title=vmexits after bootup, 
		       bar width=16pt,
       		       %%enlarge x limits={abs=0.8cm}, 		       
                       symbolic x coords={no load, cache\_pressure,forkops, fileops, hackbench, mmapops, stdout, threadops, all benchmarks},
                       xtick=data, nodes near coords, 
		       %nodes near coords align={vertical},
		       every node near coord/.append style={rotate=90, anchor=west},
                       x tick label style={rotate=45,anchor=east},]                    
	\addplot [postaction={pattern=north east lines}] coordinates {
				(no load, 130.8)
				(cache\_pressure, 141.7)
				(forkops, 119.5)
				(fileops, 146.4)
				(hackbench, 106.7)
				(mmapops, 106.6)
				(stdout, 142.7)
				(threadops, 169.1)				
				(all benchmarks, 114.1)				
			    };

	\addplot [postaction={pattern=dots}] coordinates {
				(no load, 18.2)
				(cache\_pressure, 26.6)
				(forkops, 12.6)
				(fileops, 14.3)
				(hackbench, 10)
				(mmapops, 8.1)
				(stdout, 14.9)
				(threadops, 16.4)				
				(all benchmarks, 10.2)				
			    };

\legend{worst-case, average}
\end{axis}

\end{tikzpicture}
\end{center}
\ifreport
\caption{Interrupt Latency in virtual setup}
\fi
\label{plot-virtual-overhead}
\end{figure}

\begin{figure}[!htb]
\begin{center}
\begin{tikzpicture}


\begin{axis} [name=plot1, height=8cm, width=15cm, ybar=2pt, enlarge y limits={upper,value=0.3},  %enlargelimits=0.3, 
			ymin=0,
			legend pos=north east, legend columns=-1,
                       %legend style={at={(0.5,-0.1)}, anchor=north, legend columns=-1}, 
                       ylabel={worst-case latency(us)}, %title=no pollution from gp-guest, 
		       bar width= 16pt,
			%symbolic x coords={without VPID, with VPID},
                       symbolic x coords={no load, cache pressure,forkops, fileops, hackbench, mmapops, stdout, threadops, all benchmarks},
                       xtick=data, nodes near coords, nodes near coords align={vertical}, x tick label style={rotate=45,anchor=east}, ]                    
	\addplot [postaction={pattern=north east lines}] coordinates {
				(no load, 67.6)				
				(cache pressure, 73.7)
				(forkops, 25.9)
				(fileops, 24.7)
				(hackbench, 9.3)
				(mmapops, 23.4)
				(stdout, 25.8)
				(threadops, 12.4)
				(all benchmarks, 19.8)
			      };

	\addplot [fill=black!50] coordinates {
				(no load, 130.8)				
				(cache pressure, 141.7)
				(forkops, 119.5)
				(fileops, 146.4)
				(hackbench, 106.7)
				(mmapops, 106.6)
				(stdout, 142.7)
				(threadops, 169.1)
				(all benchmarks, 114.1)
			      };
\legend{native setup, virtual setup}
\end{axis}


\begin{axis} [name=plot2, at={($(plot1.west)+(0,-12cm)$)}, height=8cm,  width=15cm, ybar=2pt, enlarge y limits={upper,value=0.2},  %enlargelimits=0.3, 
			ymin=0,
                       legend pos=north east, legend columns=-1,
			%legend style={at={(0.5,-0.1)}, anchor=north, legend columns=-1}, 
                       ylabel={average-case latency (us)}, %title=no pollution from gp-guest, 
		       bar width= 16pt,
                       %symbolic x coords={without VPID, with VPID},
			symbolic x coords={no load, cache pressure,forkops, fileops, hackbench, mmapops, stdout, threadops, all benchmarks},
                       xtick=data, nodes near coords, nodes near coords align={vertical}, x tick label style={rotate=45,anchor=east}, ]                    
	\addplot [postaction={pattern=north east lines}] coordinates {
				(no load, 44.2)				
				(cache pressure, 43.1)
				(forkops, 6.1)
				(fileops, 6.1)
				(hackbench, 4.5)
				(mmapops, 4.4)
				(stdout, 6.1)
				(threadops, 5)
				(all benchmarks, 4.9)
			      };
	\addplot [fill=black!50] coordinates {
				(no load, 18.2)				
				(cache pressure, 26.6)
				(forkops, 12.6)
				(fileops, 14.3)
				(hackbench, 10)
				(mmapops, 8.1)
				(stdout, 14.9)
				(threadops, 16.4)
				(all benchmarks, 10.2)
			      };
\legend{native setup, virtual setup}
\end{axis}

\end{tikzpicture}
\end{center}
\caption{Camparison of worst-case and average-case latency in native and virtual setup}
\label{plot-native-vs-virtual}
\end{figure}


Figure \ref{plot-native-vs-virtual} plots comparison of worst-case and average-case latency in native and virtual setup.
The worst-case interrupt latency of virtualized rt-guest has increased more than twice for all microbenchmarks.
However, considering the worst-case scenario for native setup, the increase is about $2$ times when loaded with \mcachepressure{} benchmark.
Worst-case latency is affected more than the average-case latency in a virtualized environment.
An anomaly has been observed for average-case latency without load and with \mcachepressure{}. 
In virtual setup, the average latency in these cases has decreased, however, it is still larger than average interrupt latency with remaining load conditions.
A probable reason for this anomaly is due to the cache coherence protocol. 
The cache coherence protocol can prevent eviction of cache blocks in situations when a cache block is shared between the cores by changing the block ownership. The next time data item is requested by the core, the request can be served from private cache of the other core. This is a much faster operation than fetching data from main memory. Cache coherence was inactive in the native setup as SMP was disabled to create test setup that resembles rt-guest from CPU resource perspective.

On average virtualization has increased interrupt latency by an offset of $100\mu{}s$.
The increased interrupt response time demands an in-depth investigation of each overhead component which is covered in section \ref{sec:overheadcomp}.
Figure \ref{plot-cdf-native-vs-virtual} plots CDF (cumulative density function) of latency in native and virtual setups for three load conditions.
CDF plots for rest of the load conditions are available in Appendix \ref{app:c}.
The CDF in virtual setup almost always follows the native pattern.
The patterns reveal that in virtual setup the interrupt latency is shifted by an offset which is approximately the difference in average-case latency.
The probability of worst-case interrupt latency is very low and most events converge to the average-case.

\begin{figure}[!htb]
\begin{center}

\begin{tikzpicture}

\begin{axis}[name=plot1, height=\figheightVOCDF, width=\figwidthVOCDF, ymin=0, ymax=1, xmin=0, 
		legend pos=south east,
		xlabel=latency ($\mu{}s$), 
		ylabel=probability, 
		title style={yshift=-2ex},
		title=load: \mthreadops{},
		]
	\addplot [very thick, blue, dashed] table[x=Latency,y=Probability] {./figures/native_threadops_cdf.dat};	
	\addplot [thick, red] table[x=Latency,y=Probability] {./figures/virt_threadops_10k_cdf.dat};
	\legend {\small{native (avg 5, max 12.4)}, \small{virtual (avg 16.4, max 169.1)}}
\end{axis}
%%%%%%%%%%%%%%%%%%for defense presentation%%%%%%%%%%%%%%%%%%%%
\ifdefense

\end{tikzpicture}
\end{center}
\label{plot-cdf-native-vs-virtual1}
\end{figure}
	
\begin{figure}[!htb]
\begin{center}
\begin{tikzpicture}

\fi
%%%%%%%%%%%%%%%%%%%%%ENDIF%%%%%%%%%%%%%%%%%%%%%%%%%%%%%%%%%
\begin{axis}[name=plot2, \atCmdVOCDFLowerPlotA, height=\figheightVOCDF, width=\figwidthVOCDF, ymin=0, ymax=1, xmin=0, 
		legend pos=south east,
		xlabel=latency ($\mu{}s$), 
		ylabel=probability, 
		title style={yshift=-2ex},
		title=load: \mforkops{},
		]
	\addplot [very thick, blue, dashed] table[x=Latency,y=Probability] {./figures/native_forkops_10k_cdf.dat};	
	\addplot [thick, red] table[x=Latency,y=Probability] {./figures/virt_forkops2_10k_cdf.dat};
	\legend {\small{native (avg 6.1, max 25.9)}, \small{virtual (avg 12.6, max 119.5)}}
\end{axis}

%%%%%%%%%%%%%%%%%%for defense presentation%%%%%%%%%%%%%%%%%%%%
\ifdefense

\end{tikzpicture}
\end{center}
\label{plot-cdf-native-vs-virtual2}
\end{figure}
	
\begin{figure}[!htb]
\begin{center}
\begin{tikzpicture}

\fi
%%%%%%%%%%%%%%%%%%%%%ENDIF%%%%%%%%%%%%%%%%%%%%%%%%%%%%%%%%%

\begin{axis}[name=plot3, \atCmdVOCDFLowerPlotB, height=\figheightVOCDF, width=\figwidthVOCDF, ymin=0, ymax=1, xmin=0, 		
		xlabel=latency ($\mu{}s$), 
		ylabel=probability,
		title style={yshift=-2ex},
		legend pos=south east,
		title=load: \mcachepressure{},
		]
	\addplot [very thick, blue, dashed] table[x=Latency,y=Probability] {./figures/native_cachepressure_10k_cdf.dat};
	\addplot [thick, red] table[x=Latency,y=Probability] {./figures/virt_cachepressure_10k_cdf.dat};
	\legend {\small{native (avg 43.1, max 73.7)}, \small{virtual (avg 18.2, max 131.2)}}
\end{axis}

\end{tikzpicture}
\end{center}
\ifreport
\caption{Comparison of interrupt latency CDF between native and virtual setup for different load conditions}
\label{plot-cdf-native-vs-virtual}
\else
\label{plot-cdf-native-vs-virtual3}
\fi
\end{figure}

The increase in interrupt latency can be the result of sharing of resources between both hypervisor and rt-guest, and the guests.
The current virtual setup shares core private L1/L2 caches and TLB buffers between Phidias and the real-time guest. 
The LLC is shared by both guests and Phidias.
Since hypervisor and guest execute in separate address spaces and frequency context-switching can result in pollution of caches.
Therefore, it is important to investigate how often the hypervisor involvement is required by the guest code.
Next section reports guest traps recorded for the duration of an experiment with different load conditions.

\subsection{Phidias Intervention}

\begin{figure}[!htbp]
\begin{center}
\begin{tikzpicture}


	\begin{axis} [name=plot1, ymin=0, ybar=2pt, height=5cm, width=6cm, enlarge y limits={upper,value=0.2}, 
		       legend style={at={(0.5,-0.3)}, anchor=north, legend columns=-1},
               ylabel={exits for fileops},
		       bar width=8pt,       
               symbolic x coords={ext\_int, wrmsr, cpuid, crx\_access},
               %xtick=data, nodes near coords, 
		       %nodes near coords align={vertical},
    		   ymajorgrids=true, grid style = very thin,
               x tick label style={rotate=45,anchor=east},]   

		\addplot [fill=black!30] coordinates {
				(ext\_int,222767)
				(wrmsr, 581681)
				(cpuid,0)
				(crx\_access,0)
			    };
	\end{axis}


	\begin{axis} [name=plot2, at={($(plot1.south east)+(2.25cm,0)$)}, ymin=0, ybar=2pt, height=5cm, width=6cm, enlarge y limits={upper,value=0.2}, 
		       legend style={at={(0.5,-0.3)}, anchor=east, legend columns=-1},
               ylabel={exits for forkops},
		       bar width=8pt,       
		symbolic x coords={ext\_int, wrmsr, cpuid, crx\_access},
               %xtick=data, nodes near coords, 
		       %nodes near coords align={vertical},
    		   ymajorgrids=true, grid style = very thin,
               x tick label style={rotate=45,anchor=east},]
		\addplot [fill=black!30] coordinates {
				(ext\_int,140981)
				(wrmsr, 696616)
				(cpuid,0)
				(crx\_access,543490)
			    };
	\end{axis}


	\begin{axis} [name=plot3, at={($(plot1.south west)+(0,-5.5cm)$)}, ymin=0, ybar=2pt, height=5cm, width=6cm, enlarge y limits={upper,value=0.2}, 
		       legend style={at={(0.5,-0.3)}, anchor=east, legend columns=-1},
               ylabel={exits for mmapops},
		       bar width=8pt,       
               symbolic x coords={ext\_int, wrmsr, cpuid, crx\_access},
               %xtick=data, nodes near coords, 
		       %nodes near coords align={vertical},
    		   ymajorgrids=true, grid style = very thin,
               x tick label style={rotate=45,anchor=east},]
		\addplot [fill=black!30] coordinates {
				(ext\_int,45641)
				(wrmsr, 110179)
				(cpuid,0)
				(crx\_access,18902)
			    };
	\end{axis}


	\begin{axis} [name=plot4, at={($(plot3.south east)+(2.25cm,0)$)}, ymin=0, ybar=2pt, height=5cm, width=6cm, enlarge y limits={upper,value=0.2}, 
		       legend style={at={(0.5,-0.3)}, anchor=east, legend columns=-1},
               ylabel={exits for cachepressure},
		       bar width=8pt,       
               symbolic x coords={ext\_int, wrmsr, cpuid, crx\_access},
               %xtick=data, nodes near coords, 
		       %nodes near coords align={vertical},
    		   ymajorgrids=true, grid style = very thin,
               x tick label style={rotate=45,anchor=east},]
		\addplot [fill=black!30] coordinates {
				(ext\_int,2319)
				(wrmsr, 4540)
				(cpuid,5358)
				(crx\_access,2208)
			    };
	\end{axis}


	\begin{axis} [name=plot5, at={($(plot3.south west)+(0,-5.5cm)$)}, ymin=0, ybar=2pt, height=5cm, width=6cm, enlarge y limits={upper,value=0.2}, 
		       legend style={at={(0.5,-0.3)}, anchor=east, legend columns=-1},
               ylabel={exits for hackbench},
		       bar width=8pt,       
               symbolic x coords={ext\_int, wrmsr, cpuid, crx\_access},
               %xtick=data, nodes near coords, 
		       %nodes near coords align={vertical},
    		   ymajorgrids=true, grid style = very thin,
               x tick label style={rotate=45,anchor=east},]
		\addplot [fill=black!30] coordinates {
				(ext\_int,31602)
				(wrmsr, 1731800)
				(cpuid,0)
				(crx\_access,1698102)
			    };
	\end{axis}


	\begin{axis} [name=plot6, at={($(plot5.south east)+(2.25cm,0)$)}, ymin=0, ybar=2pt, height=5cm, width=6cm, enlarge y limits={upper,value=0.2}, 
		       legend style={at={(0.5,-0.3)}, anchor=east, legend columns=-1},
               ylabel={exits for threadops},
		       bar width=8pt,       
               symbolic x coords={ext\_int, wrmsr, cpuid, crx\_access},
               %xtick=data, nodes near coords, 
		       %nodes near coords align={vertical},
    		   ymajorgrids=true, grid style = very thin,
               x tick label style={rotate=45,anchor=east},]
		\addplot [fill=black!30] coordinates {
				(ext\_int,133559)
				(wrmsr, 336922)
				(cpuid,0)
				(crx\_access,0)
			    };
	\end{axis}


	\begin{axis} [name=plot7, at={($(plot5.south west)+(0,-5.5cm)$)}, ymin=0, ybar=2pt, height=5cm, width=6cm, enlarge y limits={upper,value=0.2}, 
		       legend style={at={(0.5,-0.3)}, anchor=east, legend columns=-1},
               ylabel={exits for stdout},
		       bar width=8pt,       
               symbolic x coords={ext\_int, wrmsr, cpuid, crx\_access},
               %xtick=data, nodes near coords, 
		       %nodes near coords align={vertical},
    		   ymajorgrids=true, grid style = very thin,
               x tick label style={rotate=45,anchor=east},]
		\addplot [fill=black!30] coordinates {
				(ext\_int,187686)
				(wrmsr, 749643)
				(cpuid,0)
				(crx\_access,0)
			    };
	\end{axis}


	\begin{axis} [name=plot8, at={($(plot7.south east)+(2.5cm,0)$)}, ymin=0, ybar=2pt, height=5cm, width=6cm, enlarge y limits={upper,value=0.2}, 
		       legend style={at={(0.5,-0.3)}, anchor=east, legend columns=-1},
               ylabel={exits for noload},
		       bar width=8pt,       
               symbolic x coords={ext\_int, wrmsr, cpuid, crx\_access},
               %xtick=data, nodes near coords, 
		       %nodes near coords align={vertical},
    		ymajorgrids=true, grid style = very thin,
               x tick label style={rotate=45,anchor=east},]
		\addplot [fill=black!30] coordinates {
				(ext\_int,133559)
				(wrmsr, 336922)
				(cpuid,0)
				(crx\_access,0)
			    };
	\end{axis}

\end{tikzpicture}
\end{center}
\caption{Exit frequency with various load conditions for 30s duration}
\label{plot-phidias-inter}
\end{figure}


A key aspect of lower virtualization overhead is to keep hypervisor intervention to a minimum and allow guest code to execute directly on the processor.
This section records frequency at which guest is trapped to the hypervisor.
A small instrumentation code was inserted in hypervisor exit handler routine to measure the number of exits for the duration of an experiment. 
Figure \ref{plot-phidias-inter} plots frequency of exits during each experiment. 
Execution of an experiment with a load condition took about $30s$.

\begin{table}[!h]
\centering
\begin{tabular}{|r|c|c|}  
	\hline
    \textbf{Benchmark}	&	\textbf{total exits in $30s$}	&	\textbf{exits in $1ms$} (average) \\ \hline
	\mcachepressure{}	&		16987	&	1 	\\ \hline
	\mforkops{}	&			1381087	&	46 		\\ \hline
	\mfileops{}	&			804448	&	27 		\\ \hline
	\mhackbench{}	&		3461504	&	\textbf{115} \\ \hline
	\mmmapops{}	&			174722	&	6		 \\ \hline
	\mstdout{}	&			937329	&	31 		\\ \hline
	\mthreadops{}	&		470481	&	16 		\\ \hline
\end{tabular}
\caption{Exit frequency}
\label{exit-frequency}
\end{table}

The plot categorizes exits on the basis of exit reason. 
The guest code is trapped to hypervisor more often than the rate at which external interrupts are sent to the real-time guest.
The summary of exits is provided in Table \ref{exit-frequency}.
In worst-case guest is trapped 115 times in 1ms for experiment with the \mhackbench{} load. About one time in best-case with \mcachepressure{} load.
When the rt-guest is trapped the hypervisor emulates the desired guest behavior. The emulation takes time and could cause pollution of shared resources like caches and TLBs. Emulation and resource contention is the main reason for the increase in the interrupt response time.
Arrival of external interrupt, access to CR (control register) and write to an MSR(model specific register) are the most common reasons for exit.
In the next section, each component of virtualization overhead is measured separately.


\subsection{Overhead Components} \label{sec:overheadcomp}
The virtualization overhead consists of four components. 
Figure \ref{fig-virt-overhead-all-comp} describes all overhead components.
First is the context switch times for vmexits and vmentries.
This overhead is experienced by every guest trap and it is purely hardware overhead.
Hardware-assisted virtualization extensions support context store and restore on vmexit and vmentry transitions.
The second overhead component is hypervisor run-time, which is the emulation of desired behavior or routing of interrupts when the guest is trapped.
This component can range from few machine cycles to thousands when no action is required, to full device emulation.
The third component of the overhead is resource contention of shared resources like TLBs and LLC cache.
This component is variable and depends on how much the guest and hypervisor cached entries interfere with each other.
The last component is MMU virtualization which includes nested page-walks. 
As explained earlier in section \ref{sec:vmmu}, the virtualized system uses nested page-tables to translate 
guest physical page numbers to machine physical page numbers. The nested paging involves a higher number of pointer dereferences, which in turn
can significantly increase the memory access times for the guest code.
Phidias runtime and context-switch times can be measured by adding small instrumentation code in the hypervisor and real-time guest. 

Overheads resulting from resource contention and MMU virtualization are hard to measure.
However, the presence of the overhead due to resource contention is a fact and hardware techniques can help to reduce them.
The next chapter will shed light on some of the techniques to lower resource-contention overhead.
Here we will present results of efforts made to measure Phidias run-time and context-switch times.

\begin{figure}[!htb]
\begin{center}
\begin{tikzpicture}

\newcommand\yh{0.85cm}

%\draw[step=1cm, gray, very thin, dotted] (0,0) grid (13,2);
%\draw[black, thick, dotted] (0,0.9) -- (13,0.9) node at (13,0.9) [below] {};

\node at (0,0.5) [rectangle, draw=black, thick, fill=Blue!10, rounded corners, minimum height = \yh, minimum width = 2cm, anchor=south west] (gcbefore) {\small{guest code}};
\node at (2.1,-0.5) [rectangle, draw=black, thick, fill=Cyan!20, rounded corners, minimum height = \yh, minimum width = 1.5cm, anchor=south west] (vmexit) {\small{vmexit}};
\node at (3.7,0) [rectangle, draw=black, thick, fill=SpringGreen!30, rounded corners, minimum height = \yh, minimum width = 3.5cm, anchor=south west] (phidias) {\small{Hyp. vmexit handler}};
\node at (7.3,-0.5) [rectangle, draw=black, thick, fill=Cyan!20, rounded corners, minimum height = \yh, minimum width = 1.5cm, anchor=south west] (vmentry) {\small{vmentry}};

\node at (8.9,-0.5) [rectangle, draw=black, thick, fill=red!20, rounded corners, minimum height = \yh, minimum width = 2cm, text width = 1.7cm, anchor=south west] (rc) {\small{resource\\ contension}};

\node at (11.0,0.5) [rectangle, draw=black, thick, fill=Blue!10, rounded corners, minimum height = \yh, minimum width = 2cm, anchor=south west] (gcafter) {\small{guest code}};

\draw[black, thick, decorate, decoration={random steps,segment length=3pt,amplitude=0.5pt}] (2.05,-0.5) -- (2.05, 2.25)  
				node [above, align=center, text width = 4cm] {external event\\ or sensitive instr.\\ (trap)};
\draw[black, thick, decorate, decoration={random steps,segment length=3pt,amplitude=0.5pt}] (10.95,-0.5) -- (10.95, 2.25)  
				node [above, align=center, text width = 4cm] {guest resumes};

\draw[black, thick] (3.65,-0.5) -- (3.65, 0);
\draw[black, thick] (7.25,-0.5) -- (7.25, 0);
\draw[black, thick, dotted, <-] (3.65, -0.25) -- (4.25, -0.25); 
\draw[black, thick, dotted, ->] (6.75, -0.25) -- (7.25, -0.25);
\node at (5.45, -0.25) {Phidias runtime};

\node at (5.5, 2) (cslabel) [align=center] {context switch time};

\begin{scope}[>=latex]
	\draw [black, ->] (cslabel.west) to [bend right=25]  (vmexit.north);
	\draw [black, ->] (cslabel.east) to [bend left=25]  (vmentry.north);
\end{scope}

\end{tikzpicture}
\end{center}
\ifreport
\caption{Virtualization overhead components: Phidias runtime, context-switch time and resource contention}
\fi
\label{fig-virt-overhead-all-comp}
\end{figure}


\subsubsection{Phidias Runtime}
Phidias runtime is the time to emulate guest behavior for a trap. 
Phidias runtime can significantly influence interrupt latency and could be a major component of virtualization overhead.
Figure \ref{fig-virt-overhead-all-comp} shows Phidias runtime as the virtualization overhead component.

To measure Phidias runtime, a small instrumentation code was added at the beginning and end of Phidias vmexit handler.
The instrumentation code takes time-stamp at the beginning and end to measure worst-case vmexit handling time.
Furthermore, it also measures the number of instructions executed for worst-case execution path using x86 performance monitoring core.
Figure \ref{plot-phidias-runtime} plots histogram of worst-case Phidias runtime to handle an exit. 

\begin{figure}[!htb]
\begin{center}
\begin{tikzpicture}

\begin{axis} [ ymin=0, ybar=2pt, height=7.5cm, width=\figwidth, enlarge y limits={upper,value=0.2}, 
		       legend style={at={(0.5,-0.3)}, anchor=north, legend columns=-1},
               ylabel={Worst-Case PHIDIAS runtime ($\mu{}s$)},
		       bar width=24pt,
  		       %%enlarge x limits={abs=0.8cm}, 		       
               symbolic x coords={no load, cache\_pressure,forkops, fileops, hackbench, mmapops, stdout, threadops, all benchmarks},
               xtick=data, nodes near coords, 
		       nodes near coords align={vertical},
		       %every node near coord/.append style={rotate=90, anchor=west},
               x tick label style={rotate=25, anchor=east, xshift=4pt, font=\ttfamily\small}, ]    

			\addplot [fill=NavyBlue!50] coordinates {				
				(cache\_pressure, 105.1)
				(forkops, 103.4)
				(fileops, 115.6)
				(hackbench, 103.7)
				(mmapops, 108)
				(stdout, 109.7)
				(threadops, 117.8)				
			    };
%\legend{}
\end{axis}

\end{tikzpicture}
\ifreport
\caption{Phidias runtime under different load conditions}
\fi
\end{center}
\label{plot-phidias-runtime}
\end{figure}


\begin{table}[!h]
\centering
\begin{tabular}{|c|c|c|c|c|}  
\hline
						& \multicolumn{2}{|c|}{\textbf{Phidias runtime}} & \textbf{Instructions}	& \textbf{CPI}\\
\textbf{Benchmark}		& average ($cc$)	&	worst-case ($cc$)	&	\textbf{retired}  (worst-case) &	(worst-case) \\ \hline 
\mcachepressure{}	& 16548	& 178716 & 2322 & 77\\ \hline
\mforkops{}		& 9263 & 175702 & 2301 & 76.4\\ \hline
\mfileops{}		& 698 & 196490 & 12640 & 15.5\\ \hline
\mhackbench{} 		& 23042 & 176205 & 2301 & 76.6\\ \hline
\mmmapops{} 		& 7604 & 183566 & 2364 & 77.7\\ \hline
\mstdout{} 			& 1240 & 186435 & 21429 & 8.7\\ \hline
\mthreadops{} 		& 1265 & 200313 & 19804 & 10.1\\ \hline
\end{tabular}
\caption{Phidias worst-case execution time and CPU performance} 
\label{phidias-runtime}
\end{table}

The Phidias runtimes are taken from profiling information gathered during repeating the latency measurement experiments.
Table \ref{phidias-runtime} summarizes average and worst-case execution times of Phidias. It includes the number of
instructions executed during the worst-case path. To evaluate processor performance during the worst-case path CPI (cycles per instruction)
is calculated and enlisted in the table. 
In worst-case hypervisor code has CPI of $78$ clock cycles.
All paths in Phidias exit handlers are kept short, which is obvious from the profiled number of instructions executed when the worst-case happens.
Except for UART IO operations, there is no synchronization primitive that might delay execution of the exit handler.
Detailed results are included in Appendix \ref{app:b}.
The results shows that worst-case Phidias runtime to handle an IO operation is not different from rest of the exit handlers. 
Less than $2\%$ clock cycles are used for issuing instructions, rest in waiting for instruction or data memory from memory and caches.
For a modern x86 microarchitecture featuring superscalar processors, write-back buffers, data prefetching the measured CPI value is very high.
The most legitimate reason for poor CPI value seems to be frequent cache misses resulting in data requests satisfied from main memory, which 
can typically consume thousands of clock cycles.

\subsubsection{Context-Switch time} \label{sec:context-switch-overhead}
The context-switch time consists of hardware time needed for vmexit and vmentry.
In order to measure this overhead, a sensitive instruction was added to rt-guest code to cause a trap. 
The instruction was wrapped in assertion and deassertion of an output pin, generating a pulse of the duration of instruction execution time.
Phidias was extended to return control back to guest code when the trap is caused by the added instruction.
External measurement setup explained earlier was used to measure context switch times for ten thousand traps.
The average and worst-case times are summarized in table \ref{context-switch-time}.

\begin{table}[!h]
\centering
\begin{tabular}{|c||c|c|}  
\hline
	\textbf{Context-switch time}	&	average-case ($\mu{}s$)	&	worst-case ($\mu{}s$) \\ \hline \hline
	\mnoload{}				&	1				&		17.2 \\ \hline
	\mcachepressure{}		&	1				&		86.8 \\ \hline
\end{tabular}
\caption{Context-switch overhead (hardware time to perform vmexit and vmentry)} 
\label{context-switch-time}
\end{table}

Context-switch time was measured in two scenarios. When there is no load on the guest
the worst-case context switch time is $17.2\mu{}s$. 
The experiment was repeated with \mcachepressure{} microbenchmark executed on the same rt-guest.
In worst-case, this time can be as high as $86.8\mu{}s$ which is very close to virtualization overhead measured in section \ref{sec:virtual-overhead}.
These results validate that most part of the virtualization overhead is made up of contention on the caches shared between guest and hypervisor.
Figure \ref{plot-cdf-contextswitch} plots CDF of context-switch times without load and with cache pressure.
The CDF is generated for 10,000 samples and shows that probability of high context-switch times than average values is very small.

\begin{figure}[!htb]
\begin{center}

\begin{tikzpicture}


\begin{axis}[name=plot1, height=8cm, width=12cm,
		legend pos=south east,
		xlabel=context-switch overhead ($\mu{}s$), 
		ylabel=Probability, enlargelimits=0.05,
		%ymajorgrids=true, grid style = very thin,
		]
	\addplot [ultra thick, blue] table[x=Latency,y=Probability] {./figures/vmexitentry_10k_cdf.dat};
	\addplot [very thick, red, dashed] table[x=Latency,y=Probability] {./figures/vmexitentry_cachepressure_10k_cdf.dat};
	\legend {\small{no load (max 17.2)}, \small{cache pressure (max 86.8)}}
\end{axis}

\end{tikzpicture}
\end{center}
\ifreport
\caption{CDF of context-switch overhead}
\fi
\label{plot-cdf-contextswitch}
\end{figure}


In summary, the real-time guest suffers from increased interrupt response times in virtualized environment.
The virtualization overhead consists of Phidias runtime to handle an exit, context-switch time and contention on shared resources. 
While Phidias runtime and context-switch times can be measured separately using instrumentation code, it is hard to isolate resource contention.
The resource contention is distributed over Phidias runtime, context-switch time, and guest kernel runtime to schedule the interrupt handler.
The measured values of Phidias runtime and context-switch times appear to be affected mainly by resource contention.
Some technique to reduce resource contention are discussed and evaluated in the next chapter.


\subsection {VT-x Properties and Real-Time Guests}
This section presents some of the properties of x86 architecture that lead to non-deterministic interrupt response times of a real-time guest.

The x86 hardware typically includes big multi-level caches with high associativity. These caches perform very well in compute-extensive workloads to improve the average-case execution time by reducing the miss rate.
However, big multi-level caches suffer from long access times and miss penalties. 
Missing a data item in the cache might add the overhead of thousands of clock cycles \cite{hennessy2011computer}. 
The overhead includes access times of caches at all levels and fetching the missed data item from the main memory.
The real-time workloads are typically smaller than other computing loads and small and simpler caches might be favorable in these situations.

VT-x extensions use big data structures to manage virtual machines (e.g. VMCS and virtual APIC page).
These data structures are accessed by hardware on every VMX transition and they share the caches with guest and hypervisor code. 
Missing an item from these structures could increase the context-switch time significantly, which has been observed in section \ref{sec:context-switch-overhead}. 

Furthermore, a typical x86 microarchitecture has three levels of caches. The L3 cache is shared and the L1/L2 cache is private to each core.
There are separate L1 caches for program instruction and data.
Although recent improvements to x86 architectures enable partitioning of L3 and or L2 cache, the first level cache for data and instruction is shared by the host, guest and data structures that control VM. This is also true for multi-level TLB buffers present in x86 architecture.
The hardware support for partitioning core resources could be a huge benefit for real-time virtualization solutions. 

VT-x supports new instruction \mINVVPID{} that enables the host to emulate page invalidation requests generated by the guest.
This instruction is an example when the behavior of the x86 processor is unpredictable.
The instruction can be used by the host to selectively flush address translations cached in TLB.
The Intel SDM \cite{intel-sdm-vol3} states in certain situations this instruction can flush the whole TLB. 
The unpredictable behavior of instruction like these could lead to non-deterministic response times for the real-time guests.
