\chapter*{Zusammenfassung}

Eingebettete Systeme sind tief in unsere Gesellschaft integriert, daher neigen wir dazu ihre Existenz nicht mehr wahr zu nehmen.
Diese Systeme verbesserten die Lebensqualit{\"a}t und eine erh{\"o}hten die menschliche Produktivit{\"a}t durch Automatisierung.
Viele Anwendungen, die auf diesen Ger{\"a}ten ausgef{\"u}hrt werden, haben Echtzeitanforderungen, die strengen Timing-Einschr{\"a}nkungen unterliegen.
Echtzeitanwendungen waren fr{\"u}her einfach und liefen normalerweise auf Single-Core-Prozessoren, dies ist heutzutage nicht mehr der Fall.

Aktuelle Trends wie \emph{"Industrie 4.0"} und \emph{"Internet of Things"} f{\"u}hren zu einer vermehrten Vernetzung von Echtzeit-F{\"a}higen Systemen.
Diese Entwicklung hat zu einer erh{\"o}hten Komplexit{\"a}t des Software-Stacks gef{\"u}hrt und verlangt die Entwicklung neuer
Mechanismen, um die Systemzuverl{\"a}ssigkeit zu erreichen.
Desweiteren haben Fortschritte in der Multicore-Prozessortechnologie ihren Weg in eingebetteten Systemen gefunden.
Systemdesigner haben sich f{\"u}r Multicore-Technologie entschieden und es laufen so viele Echtzeitanwendungen auf Multicore-Architekturen wie nie zuvor.
Die Multicore-Technolo-gie hat neue M{\"o}glichkeiten er{\"o}ffnet.
Anwendungen gemischter Kritikalit{\"a}t, die fr{\"u}her auf separaten Hardwareplattformen ausgef{\"u}hrt wurden, k{\"o}nnen auf einer Plattform konsolidiert werden.
Echtzeitanwendungen k{\"o}nnen dedizierten Prozessorkernen zugewiesen werden, um eine r{\"a}umliche Isolierung zu gew{\"a}hrleisten.
Diese L{\"o}sung ist f{\"u}r Legacy-Anwendungen gleicherma{\ss}en g{\"u}ltig, sofern sich ihr Ausf{\"u}hrungsverhalten nicht {\"a}ndert.

Eine klassische L{\"o}sung zur Konsolidierung mehrerer Anwendungen auf einer Plattform ist Virtualisierung.
Seit vielen Jahren wird Virtualisierung in den Bereichen Server- und Cloud-Computing eingesetzt, um mehrere Anwendungen auf einer einzigen Maschine zu hosten.
In letzter Zeit hat sich Virtualisierung auch bei eingebettete Systeme  f{\"u}r die Anwendungskonsolidierung etabliert.
Durch die Konsolidierung k{\"o}nnen Systementwickler die Systemkosten reduzieren und die volle Rechenleistung auf einem Chip nutzen.
Allerdings sind Virtualisierungsl{\"o}sungen, die in der Unternehmensdom{\"a}ne weit verbreitet sind, nicht direkt auf Echtzeit-Systeme {\"u}bertragbar.
Diese L{\"o}sungen sollen den durchschnittlichen Durchsatz des Softwaresystems erh{\"o}hen.

Im Gegensatz dazu werden Echtzeitsysteme hinsichtlich der Antwortzeiten im worst-case Fall optimiert.
Da Echtzeitsysteme strengen zeitlichen Einschr{\"a}nkungen unterliegen muss eine L{\"o}sung f{\"u}r Echtzeitsystem eine deterministische Antwortzeit garantieren.
Eine f{\"u}r das Echtzeitsystem geeignete, zuverl{\"a}ssige Virtualisierungsl{\"o}sung muss diese Echtzeitanforderung erf{\"u}llen.
Ein Hypervisor sollte einen geringen Platzbedarf ben{\"o}tigen und muss die zeitliche und R{\"a}umliche Isolierung von virtualisierten Anwendungen gew{\"a}hrleisten.

Phidias ist ein statisch konfigurierbarer Hypervisor, der Microkernel-und Multikernel-Designmodellen folgt. 
Der Hypervisor minimiert die Komplexit{\"a}t und den Speicherbedarfdurch ein vollst{\"a}ndig statisches Design. 
W{\"a}hrend der Hypervisor Designprinzipien folgt, die f{\"u}r Echtzeitanwendungen g{\"u}nstig sind, sind seine tats{\"a}chlichen
Echtzeitf{\"a}higkeiten unbekannt. Diese Arbeit best{\"a}tigt die Echtzeitf{\"a}higkeiten von Phidias Hypervisor auf der x86-Plattform
durch Messen des Virtualisierungsoverheads f{\"u}r einen Echtzeitgast. Des Weiteren pr{\"a}sentiert die Arbeit Leistungsoptimierung
Mechanismen zur Reduzierung der Interrupt-Antwortzeiten von Echtzeit-Gast.
Die Ergebnisse zeigen, dass Latenzreduktionstechniken den Virtualisierungsaufwand deutlich reduzieren k{\"o}nnen.
