\chapter*{Abstract}
Embedded systems are deeply integrated into our society to an extent that we tend to forget about their existence.
These systems have improved quality of life and increased human productivity through automation.
Many applications that run on these devices have real-time requirements subjected to stringent timing constraints.
Real-time application used to be simple and usually ran on single core processors, but this is not true anymore. 

Recent technological trends like \emph{"Industry 4.0"} and \emph{"Internet of Things"}
has steered real-time system designers to support networking.
These improvements have led to the increased complexity of software stack and demand secure 
mechanisms to achieve system reliability.
Advancements in multicore processor technology have found its way to embedded systems. 
System designers have embraced multicore technology and more real-time application run on multicore architectures than ever before.
Multicore technology has brought a new set of opportunities.
Applications of mixed-criticality that used to run on separate hardware platforms can be consolidated on one platform.
Real-time applications can be assigned to dedicated processor core(s) to ensure spatial isolation. 
The solution is equally valid for the legacy applications, given their execution behavior
is not affected.

A classical solution to consolidate application on one platform is Virtualization.
For many years virtualization has been used in server and cloud computing to host multiple applications
on the single machine.
Recently virtualization has acquired acceptance in the embedded domain for application consolidation.
Consolidation allows system designers to reduce system costs, and utilize huge amounts of processing power available in one chip.
However, virtualization solutions that are widely accepted in enterprise domain are not applicable to the
real-time system. 
These solutions are designed to increase the average throughput of the software system.

Real-time systems are evaluated based on worst-case response times to handle an event.
Since real-time systems are subjected to stringent timing constraints, a virtualization
solution for the real-time system has to guarantee deterministic response times.
A faithful virtualization solution suitable for the real-time system has to
be aware of real-time constraints. A hypervisor should keep small footprint and 
has to ensure temporal and spatial isolation of virtualized applications.

Phidias is a statically configurable hypervisor that follows microkernel and multikernel design models.
The hypervisor keeps low complexity and memory footprint by deploying the principle of staticity.
While the hypervisor follows design principles that are favorable to real-time applications, its real-time capabilities are unknown.
This thesis confirms real-time capabilities of Phidias hypervisor on the x86 platform
by measuring virtualization overhead for a real-time guest. 
Furthermore, the thesis presents performance tuning
mechanisms to reduce interrupt response times of real-time guest. 
The results show that latency reduction techniques used can significantly reduce the virtualization overhead.
