\chapter*{Abstract}
Embedded systems are deeply integrated in our society to an extent that we tend to forget about their existence.
These systems has improved quality of life and increased human productivity through automation.
Many application that run on these devices are real-time subjected to stringent timing constraints.
Real-time application used to be simple and usually ran on single core processors, but this is not true anymore. 
%However, due to recent technological advancements it .

Recent technological trends like \emph{"Industry 4.0"} and \emph{"Internet of Things"}
has steered real-time system designers to support networking.
These improvements has led to increased complexity of software stack and demands secure 
mechanisms to achieve system reliability.
Advancements in multicore processor technology has found its way to embedded systems. 
System designers has embraced multicore technology and many real-time application run on multicore architectures than ever before.
Multicore technology has brought new set of opportunities.
Applications of mixed-criticality that used to run on separate hardware platforms can be consolidated on one platform.
Real-time applications can be assigned to dedicated processor core(s) to ensure spatial isolation. 
The solution is equally valid for legacy application, given their execution behavior
is not affected.

A classical solution to consolidate application on one platform is Virtualization.
For many years virtualization has been used in server and cloud computing to host multiple applications
on single machine.
Recently virtualization has acquired acceptance in embedded domain for application consolidation.
Consolidation allows system designers to reduce system cost and utilize huge amounts of processing power available on one chip.
However, virtualization solutions that are widely accepted in enterprise domain are not applicable to 
real-time system. 
These solutions are designed to increase average throughput of the software system.

Real-time system are evaluated based on worst-case response times to handle an event.
Since real-time systems are subjected to strict timing constraints, virtualization
solution for real-time system has to guarantee deterministic response times.
A faithful virtualization solution suitable for real-time system has to
be aware of real-time constraints. Hypervisor should keep small footprint and 
has to ensure temporal and spatial isolation of virtualized applications.

Phidias is statically configurable hypervisor that is specifically to host real-time applications.
While hypervisor footprint is small and follows design principles that are favorable to real-time applications, its
real-time capabilities are still unknown. This thesis confirms real-time capabilities of Phidias hypervisor on x86 platform
by measuring virtualization overhead for a real-time guest. Furthermore, the thesis presents performance tuning
mechanisms to reduce interrupt response times of real-time guest. 
The results shows that latency reduction techniques used can significantly reduce virtualization overhead.

%The hypervisor has to guarantee spatial 
%and time
%for real-time system are required to keep their overhead to minimum.


